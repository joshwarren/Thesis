\chapter{Descriptions of the Southern Sample}
	\label{cha:Description}
In this appendix we describe our observations as well as observation from the literature for each galaxy in our Southern sample. 

\paragraph{ESO 443-G24} has a co-rotating, kinematically-decoupled core (KDC). Both the core and the outer parts of the galaxy rotate with very low velocities ($\approx 20\,\mathrm{km\,s^{-1}}$). The [\ion{O}{iii}] emission line doublet is only detected at the centre of the galaxy, while H\,$\beta$ is only detected if the whole field of view is integrated. 

\paragraph{IC 1459} is known to contain a KDC \citep{Franx1988} embedded in a slow rotator. This is clearly seen in both the VIMOS and MUSE velocity maps (Fig. \ref{fig:stellar_vel}, \ref{fig:MUSEstellar_vel}). It is also known to have ionized gas counter rotating to the decoupled core \citep{VerdoesKleijn2000}. This is again seen by comparing Fig. \ref{fig:stellar_vel} with \ref{fig:gas_vel} and \ref{fig:MUSEstellar_vel} with \ref{fig:MUSEgas_vel}. \citet{Franz1988} claims that the gas is rotating in the same direction as the outer part of the galaxy, though this is not clear from the MUSE velocity maps. We speculate that the gas is likely to be of external origin, though propose a mechanism whereby it may have been stripped from the progenitor galaxy in the merger which resulted in the KDC and re-accreted and thus have an internal origin. The ionizing radiation field originates from the AGN and gives a LINER classification.

\paragraph{IC 1531}
% \textbf{IC 1531} seems to contain a KT. This galaxy has a very limited detection of ionized gas concentrated in the center (with the exception of [NI] (Fig. \ref{fig:NI_eqW}), which is more dispersed).

\paragraph{IC 4296}

% \textbf{IC 4296} appears to have KT, though this may be a quadrant feature. There is potentially 2 peaks in all of the gas intensity maps, one at the center of the galaxy and one to the south-east which is not seen in the image (total flux/collapsed cube).


\paragraph{NGC 612} has a large dust lane to the east of the apparent center of the galaxy and perpendicular to the axis of rotation. Most of the galaxies in our Southern sample show typical ETG behavior: old, metal rich stellar populations with high stellar velocity dispersions with the exception of NGC 612. This galaxy has very high rotational velocities for an ETG, with an extended CO disc and a younger stellar population. All this suggests that NGC 612 may have a very different history to the rest of the Southern Sample. We have shown that it joins the list of only a handful of known examples of disc-dominated radio (AGN) galaxies \citep[e.g.][]{Morganti2011}. Spiral hosts are even rarer with only 4 known cases \citep{Ledlow1998, Hota2011a, Bagchi2014, Mao2015}, though they might much more difficult to detect against the background radio emission from star formation within the spiral.

With the dust lane obscuring much of the galaxy (including its very centre) it is difficult to observed trends within the galaxy. However, the absorption line maps (other than H\,$\beta$) appear to be misaligned with the surface brightness (see Fig.\,\ref{fig:VIMOS_absorption}). The peak of the absorption line strength maps, usually at the centre of the galaxy, is instead found to the west of the peak surface brightness. The centre of the galaxy is assumed to be to the east of the peak surface brightness. Further study at higher spatial resolutions may resolve significant substructure to this galaxy. For the time being, it is clear that this galaxy is in a class of its own within our Southern sample. 


\paragraph{NGC 1316} (Fornax A) was not observed with VIMOS, however the MUSE maps show it to have a strong rotational signature, though disturbed kinematics. It has the youngest stellar population in our Southern sample ($\approx 2$\,Gyr). The ionized gas is spatially extended, but the kinematics are messy and difficult to interpret. We suggest that they are aligned with the angular moment vector of the stars, with a high velocity inflow to the southwest, almost perpendicular to the radio jet, but concede that other interpretations are possible. 

\citet{Lanz2010} observes spatially coincident radio lobes and cavities in the X-ray gas., They interpret that the different spatial sizes of the radio lobe and X-ray cavity to mean that they are from separate outbursts 0.4 and 0.1 Gyr ago, respectively.

\paragraph{NGC 1399}, the central galaxy of the Fornax Cluster \citep{Jordan2007}, is known to have kinematic twist (see MUSE map by Vaughan 2018 in prep.) on the scale of our MUSE field of view (reduced to 30"). Very little ionized gas is detected. 

As in NGC 1316 by \citet{Lanz2010}, \citet{Su2017} observes spatially coincident radio lobes and X-ray cavities. They also observe a potential third bubble which has previously inflated and become detached from the galaxy. This is a `ghost' bubble.

\paragraph{NGC 3100}
% \textbf{NGC 3100} is has NF in the stellar kinematics, however there is significant amount of ionized gas, which seems to be split into two clouds. This is most obviously seen in Fig. \ref{fig:Hbeta_eqW}. The gas also seems to have a non-standard rotation, possibly linked to its spatial distribution. 

\paragraph{NGC 3557}

% \textbf{NGC 3557} is known to be FR with very high velocities, especially considering it's size, with NF. In our maps, there some significant quadrant effects. NGC 3557 also has a very dispersed, non-centrally concentrated H$_\mathrm{\beta}$ distribution. 

\paragraph{NGC 7075}
% \textbf{NGC 7075} appears to have NF, though with quite slow velocities. There is some H$_\mathrm{\beta}$ detected at the very center of the galaxy. 

\paragraph{PKS 718-34}
% \textbf{PKS 0718-34} is a KDC, though S/N issues mean that as in IC 1459, the galaxy cannot be seen beyond the core. It has very little gas detected, though the often faint $H_\mathrm{\delta}$ line is detected. [NI] is redshifted out of the VIMOS spectral range.


