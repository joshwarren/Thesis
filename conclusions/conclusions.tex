\chapter{Discussion and Conclusion}
	\label{cha:conclusion}
Early-type galaxies (ETGs) are thought to be formed in the early universe in single cataclysmic collapses, undergoing intense star formation, before being rapidly quenched. Feedback from active-galactic nuclei (AGN) is suggested as method by which this could occur. It is becoming increasingly clear that there are two types of AGN: radiation-mode and jet-mode AGN. \citet{Heckman2014} suggest that a radiation-mode AGN is a single event in the life of a galaxy, which expels much of the gas and shuts off the star formation. Jet-mode AGN then maintain this state, stopping the re-accretion or cooling of gas, such that the host galaxy cannot restart star formation. These AGN have cycles of short active followed by longer quiescent phases. Radio galaxies (where the radio emission is not due to stellar processes such as star formation or supernova) are one marker of the active phase of Jet-mode AGN. 

To test this hypothesis, in this thesis, we have described our Southern sample of radio-selected ETGs, first constructed in \citet{Prandoni2010}. We describe our observations and observation from the literature for each galaxy within the sample in Appendix \ref{sec:Description}. 

In Chapter \ref{cha:stellar} we classify our Southern sample in to fast and slow rotators. We show that, after correcting for mass, we find a similar fraction of slow rotators to optically selected surveys. We also show that, like the general population of ETGs, most of the Southern sample are old, metal rich and alpha-element enhanced. The exceptions are NGC 612 and NGC 1316, both of which have considerably younger stellar populations ($\approx 4$ and $\approx 2$\,Gyr, respectively) that dominate their luminosities. 

In Chapter \ref{cha:gas} we investigate the ionized gas component of the interstellar medium (ISM). We first estimate the mass of \ion{H}{ii} in each galaxy, find a range from $10^4$--$10^6\,\mathrm{M_\odot}$ which is at the upper limit expected in optically-selected ETGs. This is the only suggestion we find that the sample may be intrinsically different to optically selected ETGs (other than the radio detection). Higher gas masses would be consistent with higher detection rates of different components (cold gas and dust) of the ISM \citep[e.g.][]{deRuiter2002, Leon2003, VerdoesKleijn2005}. 

Of the 11 galaxies in our Southern sample, in only 4 do we detect spatially-extended ionized gas. Of these 4, two (NGC 612 and NGC 1316) have aligned angular momentum vectors with the kinematics of the stars (although NGC 1316 could also be consistent with a misalignment). The other two (IC 1459 and NGC 3199) both show clear misalignment. NGC 3100 has particularly interesting substructures (discussed in more detail in Appendix \ref{sec:Description}), while we suggest that IC 1459 may have re-accreted the gas lost in the merger which formed the kinematically-decoupled core (KDC). In this way, gas in IC 1459 may have originated from inside the progenitor galaxy, despite it's kinematics being apparently decoupled from the rest of the galaxy. 

Finally, we use the \citet{Baldwin1981} plots, and other similar diagnostic plots such as the SAURON \citep{Sarzi2010}, H\,$\alpha$ equivalent width verse [\ion{N}{ii}]/H\,$\alpha$ \citep[WHaN2;][]{CidFernandes2011}, H\,$\alpha$ radial profile and mass--excitation \citep[MEx;][]{Nyland2016} plots in order to classify the galaxies in our Southern sample based on the dominant source of the ionizing radiation field. We find that 2 of our Southern sample (IC 1531 and NGC 3100) are classified as Seyfert 2s; 7 are LINERs, of which 4 we attribute to radiation from to the AGN. The emission lines required to classify the galaxies in this way are not detected for the remaining two galaxies (ESO 443-G24 and PKS 718-34).

All of this suggests that our radio-selected galaxies are very similar to radio-quiet ETGs in all defining characteristics accessible to study within $V$-band data. This agrees with the hypothesis that all except the most extreme radio galaxies are jet-mode AGN and that this is a normal phase in the duty cycle of all ETGs. 

Two galaxies of the sample that are potential outliers are NGC 612 and NGC 1316. Both galaxies are fast rotators, with spatially-extended, ionized gas which is a aligned kinematically to the angular momentum vector of the stars. It is well known that intrinsically axisymmetric radio galaxies are extremely rare with only a handful known \citep[e.g.][]{Heckman1982, Ledlow1998, Hota2011a, Morganti2011, Mao2015}. This suggests that while the formation of jet-mode AGN does not explicitly require a dry major merger (which results in a slow rotator and expels internal gas reservoirs), such an event significantly enhances the likelihood of it occurring. 

\section{Future Work}
	\label{sec:future}
	With this data set, future time could be spent producing dynamical Jeans Anisotropic Models (JAM) using the methods by \citet{Cappellari2008}. This would give an accurate estimation of the inclination and the mass-to-light ratio of our sample. Having said that, the VIMOS data may not be of sufficient quality, with the features due to the VIMOS quadrants causing the \textsc{JAM} routine\footnote{http://www-astro.physics.ox.ac.uk/~mxc/software/\#jam} to fail. 

	We already have observation for 9 out of 11 galaxies in the sample with the Atacama Large Millimeter Array (ALMA). The moment-0 measurements for \ce{^{12}CO(2-1)} are overlaid on all maps in this thesis, but a comparison between the kinematics of the CO and the ionized gas also needs to be made. In particularly, NGC 3100 shows interesting, non-spatially coincident structures in both the ionized gas maps and the molecular gas.

	In order to observed the third (and final) phase of the ISM, the hot (X-ray) gas, it would be useful to get X-ray observation from the \textit{Chandra Observatory}. Some observations of our sample already exist within the literature: \citet{Lanz2010} has observed NGC 1316 and \citet{Su2017} has observed NGC 1399. 

