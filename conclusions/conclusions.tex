\chapter{Discussion and Conclusion}
	\label{cha:conclusion}
The progenitors to early-type galaxies (ETGs) are thought to be formed in the early universe in single cataclysmic collapses, undergoing intense star formation, before being rapidly quenched at some point in their history. By stripping the galaxy of most of its atomic and molecular (collectively described as cold) and ionized (warm) gas reservoirs and subsequently stopping new reservoirs from cooling, feedback from active-galactic nuclei (AGN) is a suggested method by which this could occur. Indeed many semi-analytic models and numerical simulations require such feedback in order to reproduce the known properties of ETGs \citep[e.g.][]{Kauffmann2000, DiMatteo2005, Springel2005, Bower2006}. 

It is becoming increasingly clear that there are two types of AGN: radiation-mode and jet-mode AGN \citep[e.g.][]{Antonucci2012}. \citet{Heckman2014} suggest that a radiation-mode AGN is a single event in the life of a massive galaxy, which expels much of the gas and shuts off the star formation. Jet-mode AGN then maintain this state, stopping the re-accretion or cooling of gas, such that the host galaxy cannot restart star formation. These AGN have cycles of short activity followed by longer quiescent phases. Radio galaxies (where the radio emission is not due to stellar processes such as star formation or supernova) are one marker of the active phase of jet-mode AGN. If this is the case, then we would expect samples of radio galaxies to be no different to optically selected samples of ETGs. 

To test this hypothesis, in this thesis, we have described our Southern sample of radio-selected ETGs, first constructed in \citet{Prandoni2010}. We describe our observations and those of the literature for each galaxy within the sample in Appendix \ref{cha:Description}. In Chapter \ref{cha:stellar}, we observe the stellar kinematics and populations of our Southern sample. We compare known characteristics of optically-selected ETGs to our observations using the Atlas$^\text{3D}$ and MASSIVE surveys as control samples. Firstly we classify our Southern sample in to fast and slow rotators. We show that, after correcting for mass, there is a similar fraction of slow rotators in our Southern sample to the optically selected surveys. We also show that, like the general population of ETGs, most of the Southern sample are old, metal rich and alpha-element enhanced. The exceptions are NGC 612 and NGC 1316, both of which have considerably younger stellar populations ($\approx 4$ and $\approx 2$\,Gyr, respectively) that dominate their luminosities. 

In Chapter \ref{cha:gas} we observe the distribution, kinematics and ionization of the warm gas in our Southern sample. Again, this is compared to observations by the Atlas$^\text{3D}$ project. We first estimate the mass of \ion{H}{ii} in each galaxy, find a range from $10^4$--$10^6\,\mathrm{M_\odot}$ which is at the upper limit expected in optically-selected ETGs. This is the only suggestion we find that the sample may be intrinsically different to optically selected ETGs (other than the radio detection). Higher gas masses would be consistent with higher detection rates of different components (cold gas and dust) of the ISM \citep[e.g.][]{deRuiter2002, Leon2003, VerdoesKleijn2005}. In only 4 galaxies in our Southern sample, do we detect spatially-extended ionized gas. Of these, two (NGC 612 and NGC 1316) have aligned angular momentum vectors with the kinematics of the stars (although NGC 1316 could also be consistent with a misalignment). The other two (IC 1459 and NGC 3199) both show clear misalignment. NGC 3100 has particularly interesting substructures (discussed in more detail in Appendix \ref{cha:Description}), while we suggest that IC 1459 may have re-accreted the gas lost in the merger which formed the kinematically-decoupled core (KDC). In this way, the observed gas in IC 1459 may have originated from inside the progenitor galaxy, despite it's kinematics being apparently decoupled from the rest of the galaxy. Finally, we use the \citet[; BPT]{Baldwin1981} plots, and other similar diagnostic plots such as the SAURON \citep{Sarzi2010}, H\,$\alpha$ equivalent width verse [\ion{N}{ii}]/H\,$\alpha$ \citep[WHaN2;][]{CidFernandes2011}, H\,$\alpha$ radial profile and mass--excitation \citep[MEx;][]{Nyland2016} plots in order to classify the galaxies in our Southern sample based on the dominant source of the ionizing radiation field. We find that 2 of our Southern sample (IC 1531 and NGC 3100) are classified as Seyfert 2s; 7 are LINERs, 4 of which we attribute the ionizing potential to the radiation field from to the AGN. The emission lines required to classify the galaxies in this way are not detected for the remaining two galaxies (ESO 443-G24 and PKS 718-34). These are consistent with the numbers found by \citet{Nyland2016}.

All of this suggests that our radio-selected galaxies are very similar to radio-quiet ETGs in all defining characteristics accessible to study within $V$-band data. This agrees with the hypothesis that radio activity is a normal phase in the duty cycle of massive ETGs. 

The only difference that we observe is slightly higher \ion{H}{ii} masses in radio galaxies than in optically selected ETGs. This may be due to approximations made to the flux calibration, but if true, shows that radio galaxies either have more total gas or more ionized gas, suggesting a stronger ionizing radiation field from the AGN. Both scenarios might be expected if the galaxy is in the process of reheating any gas allowed to cool in quiescent phase. 

Two galaxies of the sample that are potential outliers are NGC 612 and NGC 1316. Both galaxies are fast rotators, with spatially-extended, ionized gas which is aligned kinematically to the angular momentum vector of the stars. It is well known that intrinsically axisymmetric (disc) radio galaxies are extremely rare with only a handful known \citep[e.g.][]{Heckman1982, Ledlow1998, Hota2011a, Morganti2011, Mao2015}. This suggests that while the formation of jet-mode AGN does not explicitly require a dry major merger (which destroys the disc resulting in a slow rotator and expels internal gas reservoirs), such an event significantly enhances the likelihood of it occurring. 

\section{Future Work}
	\label{sec:future}
	With this data set, we could produce dynamical Jeans Anisotropic Models (\textsc{jam}\footnote{http://www-astro.physics.ox.ac.uk/~mxc/software/\#jam}) using the methods by \citet{Cappellari2008}. This would give an accurate estimation of the inclination and the total mass of the galaxies in our sample. This would allow us to study the mass-to-light ratio as well as compare quantities of gas as a fraction of total mass. Having said that, the VIMOS data may not be of sufficient quality: the features due to the VIMOS quadrants may cause the \textsc{jam} routine to fail. 

	We already have observation for 9 out of 11 galaxies in the sample with the Atacama Large Millimeter Array (ALMA). The moment-0 measurements for \ce{^{12}CO(2-1)} are overlaid on all maps in this thesis, and further work on these observation is currently being carried out Ruffa (2018 in prep.). A comparison between the kinematics of the CO and the ionized gas should be made once this is complete. In particularly, NGC 3100 shows interesting, non-spatially coincident structures in both the ionized gas maps and the molecular gas, which seem to be direct observation of feedback from the AGN via the radio jets. 

	In order to observed the final phase of the ISM, the hot (X-ray) gas, it would be useful to get X-ray observation from the \textit{Chandra Observatory}. Some observations of our sample already exist within the literature: IC 1459 was observed by \citet{Fabbiano2003}, IC 4296 by \citet{Pellegrini2002}, NGC 1316 by \citet{Lanz2010}, NGC 3557 by \citet{Ponman2001} (first published by \citealt{Balmaverde2005}) and several observation of NGC 1399 are available \citep[e.g.][]{Su2017}. Many have already noted X-ray cavities apparently created by the radio jets as evidence for AGN feedback by these systems. The size of these cavities can be used to compute the energy imparted to the hot gas from the radio jets and by assuming some efficiency thereby estimate the power contained in the jet. Additionally, the accretion power can be probed using the X-ray luminosity of the point source (AGN) at the centre of the galaxy, which can be directly compared to the radio jet power.

	All of these systematically study a different phase of the galaxies in our Southern sample in a similar manner to the Atlas$^\text{3D}$ project. This allows direct comparisons between our radio-selected sample and the optically-selected Atlas$^\text{3D}$. 

	Finally, a major improvement would be to re-observe to survey area with the VLA to add missing galaxies to turn this sample into a volume-limited sample (currently the sample is defined by apparent radio flux density and apparent magnitude as well as redshift).
