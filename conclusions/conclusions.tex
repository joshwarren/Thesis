\chapter{Discussion and Conclusions}
	\label{cha:conclusion}
The progenitors to early-type galaxies (ETGs) are thought to form in the early universe in single cataclysmic collapses, undergoing intense star formation in the process, before being rapidly quenched a short time later in their history. One method by which this quenching could occur in massive galaxies is feedback from active galactic nuclei (AGN), whereby AGN both strip the galaxies of most of their gas (the fuel for star formation) and subsequently stop the cooling of hot gas and thus the replenishment of the gas reservoirs. Primarily, this requires removing the atomic and molecular (collectively the cold) gas, but it is observed that ionized (warm) gas is also present in only very small quantities in ETGs. In fact, many semi-analytic models and numerical simulations require such feedback to reproduce the known properties of ETGs \citep[e.g.][]{Kauffmann2000, DiMatteo2005, Springel2005, Bower2006}. 

It is becoming increasingly clear that there are two types of AGN: radiative-mode and jet-mode AGN \citep[e.g.][]{Antonucci2012}. \citet{Heckman2014} suggest that the radiative-mode AGN is a single event in the life of a massive galaxy, that expels most of the gas and halts star formation. Jet-mode AGN then maintain this state, stopping the re-accretion and/or cooling of gas, such that the host galaxy cannot replenish its gas reservoir and/or restart star formation. These jet-mode AGN have multiple short activity followed by longer quiescent phases. Radio galaxies (where the radio emission is not due to stellar processes such as star formation or supernovae) are one sign of the active phase of jet-mode AGN. If jet-mode AGN are indeed ordinary (massive) ETGs in an active phase, then we would expect samples of radio galaxies to be largely identical to optically-selected ETG samples. 

To test this hypothesis, we use a sample of radio-selected ETGs constructed by \citet{Prandoni2010} and hereafter referred to as the Southern Sample. This thesis presents integral field spectroscopy (IFS) observations of the sample galaxies from the Visible Multi-object Spectrograph (VIMOS)  and archival data from the Multi-unit Spectroscopic Explorer (MUSE), both mounted on the Very Large Telescope (VLT). Our data-reduction pipelines for data from each instrument are set out in Chapter \ref{cha:Data} and include all the standard IFS data-reduction steps as well as specific steps in the VIMOS pipeline to account for the flexure and fibre-to-fibre cross-talk of VIMOS. The VIMOS pipeline also includes ad-hoc corrections to correct the large offsets in flux in between its 4 detectors as well as a fringe-like pattern in each spectra. Both VIMOS and MUSE reduction pipelines include flux-calibration steps (although the ad-hoc corrections in the VIMOS pipeline means that its flux-calibration should be regarded as approximate). Finally all observations are transformed into cube format. We then analyse both VIMOS and MUSE datasets with the same methods which we summarise below (see Chapter \ref{cha:Data} for a more detailed description). 

Each datacube is spatially binned, by increasing the bin sizes until a target signal-to-noise ratio (S/N) of the bin is reached. Each bin is then analysed fitted using a library of empirical stellar templates convolved with a line-of-sight velocity distribution (LOSVD; here parametrised by a mean velocity and velocity dispersion). Emission lines, from the warm gas, are also fitted using positive Gaussian templates, convolved by its own LOSVD, independent of the stellar LOSVD. These steps result spatially-resolved maps of stellar kinematics and ionized gas distributions and kinematics. We also measure the absorption line strengths which are then used to find a best-fitting stellar population model. A full description of the properties of each of our sample galaxies derived from our observations (and some from the literature) is given in Appendix \ref{cha:Description}. 

In Chapter \ref{cha:stellar}, we discussed the stellar kinematics and populations of our Southern Sample galaxies. We compared known characteristics of optically-selected ETGs to those of our sample galaxies using the Atlas$^\text{3D}$ and MASSIVE surveys as control samples. Firstly we classified our Southern Sample galaxies into a number of kinematic classifications: regular/non-regular rotators, kinematic substructure groups and fast/slow rotators. We showed that, after correcting for stellar mass, there is a similar fraction of slow rotators in our Southern Sample and optically-selected samples. We also showed that, like the general population of ETGs, most of the Southern Sample galaxies are old, metal rich and alpha-element enhanced. The exceptions are NGC 612 and NGC 1316, both of which have considerably younger stellar populations ($\approx 4$ and $\approx 2$\,Gyr, respectively) that dominate their optical spectra. 

As suggested by several authors in the literature \citep[e.g.][]{Collin1999, Diamond-Stanic2012, LaMassa2013}, if cold gas is the fuel source that is accreted on to the the central black holes of our Southern sample in order to power the radio jets, unless it is accreted in a highly turbulent fashion, it might be expected that some of that gas might form stars. If this is the case then a young stellar population might be expected to dominate in the very central (1--2) spaxels. We our best-fitting age maps show no evidence of this. 

In Chapter \ref{cha:gas}, we discussed the distribution, kinematics and ionization of the warm gas in our Southern Sample galaxies. Again, these were compared to observations made by the Atlas$^\text{3D}$ project. We first estimated the mass of \ion{H}{ii} in each galaxy and found a range of $10^4$\,--\, $10^6\,\mathrm{M_\odot}$, at the upper end of the range expected from optically-selected ETGs. This is the only suggestion we find that our sample galaxies may be intrinsically different than optically-selected ETGs (other than those with significant radio emission). Higher gas masses are consistent with higher detection rates of different components (cold gas and dust) of the interstellar medium \citep[ISM; e.g.][]{DeRuiter2002, Leon2003, VerdoesKleijn2005}. We detect spatially-extended ionized gas in only 4 galaxies of our Southern Sample. Of these, just one (NGC 612) has its ionized gas angular momentum vector aligned with that of its stars, two (IC 1459 and NGC 3100) show a clear kinematic misalignment, which is interpreted as evidence that this gas was accreted onto the galaxies, rather than originating from stellar waste (e.g.\ winds and supernova) and NGC 1316 is not clear. NGC 3100 has particularly interesting ionized gas substructures, with two intensity peaks spatially coincident with the radio jets and located in the gaps in the CO ring (discussed in more detail in Appendix \ref{cha:Description}). 

Finally, in Chapter \ref{cha:gas} we also used the \citeauthor{Baldwin1981} (\citeyear{Baldwin1981}; BPT) plots and other similar diagnostics to classify the galaxies in our Southern Sample based on the dominant source of ionizing photons. These other diagnostics include the SAURON diagnostics plot \citep{Sarzi2010}, the H$\alpha$ equivalent width versus [\ion{N}{ii}]/H$\alpha$ ratio \citep[WHaN2;][]{CidFernandes2011}, the H$\alpha$ radial profile and mass\,--\,excitation \citep[MEx;][]{Nyland2016}. We found that 9 of our Southern Sample galaxies are low-ionized nuclear emission region galaxies (LINERs; 5 of which are due to the radiation field from the AGN) and one (IC 1531) is a Seyfert II. The emission lines required to classify the galaxies are not detected in the remaining galaxy (PKS 718-34; i.e.\ a passive galaxy). These numbers are consistent with the fractions found by \citet{Nyland2016} in the Atlas$^\text{3D}$ sample.

All of these results suggest that our radio-selected galaxies are very similar to radio-quiet ETGs, in all the defining characteristics possible to study with $V$-band IFS data. This supports the hypothesis that radio activity is a normal but short-lived phase (i.e.\ a duty cycle) in the life of massive ETGs. 

The only difference that we observe is slightly higher \ion{H}{ii} masses in radio galaxies than in optically-selected ETGs. This could be explained by approximations made during the flux calibration and/or gas mass calculations. However, if true, this would show that radio galaxies either have more total gas or more ionized gas, the latter suggesting a stronger ionizing radiation field from the AGN. Both scenarios are expected if the galaxy is in the process of reheating gas previously allowed to cool during the quiescent phase. 

Two galaxies of the sample that are potential outliers are NGC 612 and NGC 1316. Both galaxies are fast rotators, with young stellar populations and spatially-extended ionized gas. In NGC 612 the gas is kinematically aligned with the stars, which is consistent with an internal origin to the gas and is one of only a handful of radio galaxies with lenticular morphology \citep[e.g.][]{Heckman1982, Ledlow1998, Morganti2011}. We suggest that a merger event may have caused a burst of star formation in NGC 1316 resulting in the young stellar population, however we find no other evidence that NGC 612 has a recent merger history suggesting that radio jets do not require a merger to form. In fact as stated above, the kinematics of the gas disc in NGC 612 is consistent with an internal origin of the gas, suggesting that in rare cases, radio galaxies can fueled through entirely secular processes. 


\section{Future Work}
	\label{sec:future}
	With this data set, it should be possible to produce dynamical models using the Jeans Anisotropic Models (\textsc{jam}\footnote{\url{http://www-astro.physics.ox.ac.uk/~mxc/software/\#jam}}) method of \citet{Cappellari2008}. This would give an accurate estimation of the inclination and mass-to-light ratio of the galaxies in our sample, which would allow us to constrain the stellar and dark-matter masses, and compare the gas masses as fractions of the total masses. However, it may be that the VIMOS data are not of sufficient quality; the offsets between the VIMOS quadrants may cause the \textsc{jam} routines, which requires high quality data, to fail. 

	We already have observations of 9 out of 11 sample galaxies with the Atacama Large Millimeter/sub-millimeter Array (ALMA). The moment-0 (total intensity) maps of the \ce{^{12}CO(2-1)} line are overlaid on all maps in this thesis, and further work on these observation is currently being carried out by Ruffa et.al.\ (in prep.). A comparison of the CO and the ionized gas kinematics should be made once this is complete. In particular, NGC 3100 shows interesting, non-spatially-coincident ionized gas and molecular gas structures. This may be direct evidence of feedback from the AGN via the radio jets. 

	To constrain the final phase of the ISM, the hot (X-ray) gas, it would be useful to obtain X-ray observations from the \textit{Chandra Observatory}. Some observations of our sample galaxies already exist in the literature: IC 1459 was observed by \citet{Fabbiano2003}, IC 4296 by \citet{Pellegrini2002}, NGC 1316 by \citet{Lanz2010}, NGC 3557 by \citeauthor{Ponman2001} (\citeyear{Ponman2001}; first published by \citealt{Balmaverde2005}) and several observations of NGC 1399 are available \citep[e.g.][]{Su2017}. Many have already noted X-ray cavities apparently created by radio jets as evidence of AGN feedback in these systems. The size of these cavities can be used to estimate the energy imparted to the hot gas by the radio jets. Assuming some efficiency, we can thereby estimate the power of the jets. Additionally, the accretion power can be probed using the X-ray luminosity of the point source (AGN) at the centre of the galaxy, which can then be directly compared to the radio jet power. This allows us to assess if this method of feedback is providing sufficient power to eject/heat the ISM to maintain the dearth of star formation in the host galaxy.

	The cold gas (via \ce{^{12}CO(2-1)}) observations from ALMA, the warm gas and stellar observations from this thesis, and the proposed hot gas observations from \textit{Chandra} will together provide a systematic and exhaustive study similar to that of the Atlas$^\text{3D}$ project. This will allow for more thorough comparisons of our radio-selected sample with optically-selected samples.

	A final improvement (although perhaps unrealistic) would be to re-observe the survey area to add any missing galaxies, such that this sample becomes volume-limited (currently the sample is defined by apparent radio flux density, apparent magnitude and redshift). This would remove the bias towards high-mass galaxies that is inherent in our Southern Sample.
