\chapter{Introduction}
For millenia, Man has turned his gaze skyward in awe and wonder at the heavens and wondered at the processes that processes that drive structure on the colossal scale of galaxies and galaxy clusters. As physicists we conduct experiments to better out understanding of the processes and laws that control the universe; an old joke goes that as astronomers, our experiment has already been conducted for us, it's called The Big Bang: our job is to use it to test the theories and laws of physics. Indeed, astronomy is a unique opportunity to test our knowledge as scales and limits that are not attainable here on Earth.

% As telescopes were developed for different parts (or bands) of the electromagnetic spectrum it became clear that many objects would radiate quite brightly in some parts of the spectrum, while being completely invisible in others. Nowhere was this more evident and confusing than in radio emissions. In most parts of the spectrum, if galaxy emits in that band, images more or less follow the morphology of the galaxy as seen in the visible bands. In the radio bands, however, the brightest sources (and therefore the first to be observed), appeared as two components on either side of galaxy. 

The most successful cosmological description of the universe is the dark energy (represented by \textLambda), cold dark matter ({\textLambda}CDM) coupled with inflation. This describes how the current state of the universe (69\% Dark Energy (currently best described with a scalar field), 26\% cold dark matter (a matter-like species which only interacts gravitationally), \~5\% ordinary baryonic matter and <0.003\% radiation, complete with deep gravitational potential wells in which galaxies and galaxy clusters have formed) has evolved from an infinite and random spacetime 'foam'. Quantum fluctuation within this 'foam' cause region to form which has the corrected conditions for inflation to occur: this includes a quantum scalar field with a sufficient energy density to dominate the region. This field cause the exponential expansion of space at superluminal speeds. Quantum fluctuations are rapidly expanded to beyond the scale of the horizon (distance that a massless particle can travel within the age of the universe). At this point they become frozen in as perturbation to the density profile of the universe (any perturbations prior to the start of inflation are stretched so thin that they can be considered completely negligible). Once the energy density of the inflationary field becomes too low, inflation ends, radiation dominates and the expansion slows to subluminal speeds. As the horizon expands at the speed of light, the density perturbations re-enter the horizon (smallest first), providing seed for the gravitational collapse of dark matter. This process has hardwired hierarchical growth of structure (where the smallest structures, small galaxies, form first, which then coalesce into large galaxies, which form groups, clusters and eventually super-clusters) into the cosmos. 
 %though these fractions have varied throughout the course of the universe depending how each species is affected by and itself effects the expansion of space. 

Having said that, we have said nothing the materials that we can directly observe: baryonic matter and radiation. Baryonic matter is far more complicated than dark matter and thus when we observe galaxies we see far more structure than simply smooth halos which it is believed that dark matter mostly exists as. The large plethora of shapes of galaxies are summed up in the Hubble Diagram. However, recent studies have shown that we cannot completely classify a galaxy by its morphology only (though some broad trends with morphology do exist). A more physical representation might be by color: \citet{} showed that galaxies exist within two distinct regions (known as the blue main sequence and red cloud) on a color-color plot. These represent galaxies that are actively star forming (which are bluer) and galaxies which have had their star formation quenched (these galaxies are redder). These two classes roughly follow the Hubble diagram: spirals tend to exist in the main sequence, while elliptical and lenticular (S0s) galaxies (which are collectively known as Early-type galaxies (ETGs)) mostly exist in the red cloud. This thesis has a particular focus on ETGs for reasons described below and more a detailed description of the our current understandings of ETGs is given in section \ref{sec:introETG}. 

Given their low star formation rates, ETGs are dominated by old stars, while spirals are dominated by young stars. The reasons for this dichotomy is a major part of modern astrophysics. ETGs are often the most massive galaxies: they clearly must have undergone substantial star formation in their early history. They either must have run out of fuel for forming stars or have undergone some process which is preventing star formation. Many have been observed to contain large resevours of cold (molecular) gas, the material from which stars are formed, meaning that we must favor the latter. 

Many studies now point to the fact that most (if not all) galaxies contain a super-massive black hole (SMBH) at the center. Many studies also show that there are correlations (known as scaling relations) between the properties of the central SMBH and the host galaxy despite orders of magnitude differences between the sphere of gravitational influence of the SMBH and the size of the host galaxy. 

Many galaxies contain a bright point source at the center. These are known as Active Galactic Nuclei (AGN) and in extreme cases these can outshine the combined starlight of the host galaxy. AGN are understood to by the result of accretion of the inter-stellar medium (ISM) onto the central SMBH and the energy emitted by the AGN is generally invoked to explain the scaling relationships between the SMBHs and their host galaxies as well as the quenching of star formation. This process is known as AGN feedback. 

AGNs are very varied, however much work has been done on a single unified model of AGNs, where the differences between individual observations can be explained by differing orientations of the AGN. Despite this, it is becoming increasingly clear that AGNs exist in two different modes: Radiative and Jet-mode. 










The rest of this chapter will cover the background required to study the LERG population. Firstly, since RGs are mostly found in Early Type Galaxies (ETGs) we will cover our current understanding of ETGs. This is mostly a summary of the recent Integral Field Spectroscopy (IFS) surveys including Atlas3D, MaNGa, MASSIVE, SAMI and Califa. This can be considered a setting out of the control sample for our later investigations. After this we will give a summary of the current state of our understanding of Active Galactic Nuclei (AGN). This will include covering RGs.

\section{Early Type Galaxies}
	\label{sec:introETG}
	In this section we mostly summarize Cappellari's excellent review of IFS studies of ETGs \citep{Cappellari2016}. 



\section{Active Galactic Nuclei}
	\label{sec:introAGN}
	Much of this section, makes use of the review by \citet{Heckman2014} and the references therein, and we would point readers in the direction of this review if more detail is required. 

	Firstly, it is necessary define what we mean by an Active Galactic Nuclei (AGN). Put simply it is when the center of galaxy emits light (in any part of the spectrum) than can be simply attributed to the stars alone. The consensus is the extra emission is due to the accretion of the Inter-Stellar Medium (ISM) onto the central Super-Massive Black Hole (SMBH). It has become increasingly clear in recent years that there exists to distinct modes of AGN: radiative and jet modes. This are explored in the following sections.

	\subsection{Radiative Mode}
		\label{subsec:introRadiative}

	\subsection{Jet Mode}
		\label{subsec:introJet}
