\chapter{Introduction}
For millenia, Man has turned his gaze skyward in awe and wonder at the heavens and wondered at the processes that processes that drive structure on the colossal scale of galaxies and galaxy clusters. As physicists we conduct experiments to better out understanding of the processes and laws that control the universe; an old joke goes that as astronomers, our experiment has already been conducted for us, it's called The Big Bang: our job is to use it to test the theories and laws of physics. Indeed, astronomy is a unique opportunity to test our knowledge as scales and limits that are not attainable here on Earth.

% As telescopes were developed for different parts (or bands) of the electromagnetic spectrum it became clear that many objects would radiate quite brightly in some parts of the spectrum, while being completely invisible in others. Nowhere was this more evident and confusing than in radio emissions. In most parts of the spectrum, if galaxy emits in that band, images more or less follow the morphology of the galaxy as seen in the visible bands. In the radio bands, however, the brightest sources (and therefore the first to be observed), appeared as two components on either side of galaxy. 

The most successful cosmological description of the universe is the dark energy (represented by \textLambda), cold dark matter ({\textLambda}CDM) coupled with inflation. This describes how the current state of the universe (69\% Dark Energy (currently best described with a scalar field), 26\% cold dark matter (a matter-like species which only interacts gravitationally), \~5\% ordinary baryonic matter and <0.003\% radiation, complete with deep gravitational potential wells in which galaxies and galaxy clusters have formed) has evolved from an infinite and random spacetime 'foam'. Quantum fluctuation within this 'foam' cause region to form which has the corrected conditions for inflation to occur: this includes a quantum scalar field with a sufficient energy density to dominate the region. This field cause the exponential expansion of space at superluminal speeds. Quantum fluctuations are rapidly expanded to beyond the scale of the horizon (distance that a massless particle can travel within the age of the universe). At this point they become frozen in as perturbation to the density profile of the universe (any perturbations prior to the start of inflation are stretched so thin that they can be considered completely negligible). Once the energy density of the inflationary field becomes too low, inflation ends, radiation dominates and the expansion slows to subluminal speeds. As the horizon expands at the speed of light, the density perturbations re-enter the horizon (smallest first), allowing them to collapse gravitationally. This process has hardwired hierarchical growth of structure (where the smallest structures, small galaxies, form first, which then coalesce into large galaxies, which form groups, clusters and eventually super-clusters) into the cosmos. 


 though these fractions have varied throughout the course of the universe depending how each species is affected by and itself effects the expansion of space. 










The rest of this chapter will cover the background required to study the LERG population. Firstly, since RGs are mostly found in Early Type Galaxies (ETGs) we will cover our current understanding of ETGs. This is mostly a summary of the recent Integral Field Spectroscopy (IFS) surveys including Atlas3D, MaNGa, MASSIVE, SAMI and Califa. This can be considered a setting out of the control sample for our later investigations. After this we will give a summary of the current state of our understanding of Active Galactic Nuclei (AGN). This will include covering RGs.

\section{Early Type Galaxies}
	\label(sec:introETG)
	In this section we mostly summarize Cappellari's excellent review of IFS studies of ETGs \citep{Cappellari2016}. 



\section{Active Galactic Nuclei}
	\label{sec:introAGN}
	Much of this section, makes use of the review by \citet{Heckman2014} and the references therein, and we would point readers in the direction of this review if more detail is required. 

	Firstly, it is necessary define what we mean by an Active Galactic Nuclei (AGN). Put simply it is when the center of galaxy emits light (in any part of the spectrum) than can be simply attributed to the stars alone. The consensus is the extra emission is due to the accretion of the Inter-Stellar Medium (ISM) onto the central Super-Massive Black Hole (SMBH). It has become increasingly clear in recent years that there exists to distinct modes of AGN: radiative and jet modes. This are explored in the following sections.

	\subsection{Radiative Mode}
		\label{subsec:introRadiative}

	\subsection{Jet Mode}
		\label{subsec:introJet}
