\chapter{Introduction}
For millenia, Man has turned his gaze skyward in awe and wonder at the heavens and wondered at the processes that processes that drive structure on the colossal scale of galaxies and galaxy clusters. As physicists we conduct experiments to better out understanding of the processes and laws that control the Universe; an old joke goes that as astronomers, our experiment has already been conducted for us, it's called The Big Bang, and our job is to use it to test the theories and laws of physics. Indeed, astronomy is a unique opportunity to test our knowledge as scales and limits that are not attainable here on Earth.

As telescopes were developed for different parts of the electromagnetic spectrum it became clear that many objects would radiate quite brightly in some parts of the spectrum, while being completely invisible in others. Nowhere was this more evident and confusing that in the radio. Radio galaxies often had two components each offset to opposite side of the galaxy.




The rest of this chapter will cover the background required to study the LERG population. Firstly, since RGs are mostly found in Early Type Galaxies (ETGs) we will cover our current understanding of ETGs. This is mostly a summary of the recent Integral Field Spectroscopy (IFS) surveys including Atlas3D, MaNGa, MASSIVE, SAMI and Califa. This can be considered a setting out of the control sample for our later investigations. After this we will give a summary of the current state of our understanding of Active Galactic Nuclei (AGN). This will include covering RGs.

\section{Early Type Galaxies}
	\label(sec:introETG)
	In this section we mostly summarise Cappellari's excellent review of IFS studies of ETGs \citep{Cappellari2016}. 



\section{Active Galactic Nuclei}
	\label{sec:introAGN}
	Much of this section, makes use of the review by Heckman \citep{Heckman2014} and the references therein, and we would point readers in the direction of this review if more detail is required. 

	Firstly, it is necessary define what we mean by an Active Galactic Nuclei (AGN). Put simply it is when the center of galaxy emits light (in any part of the spectrum) than can be simply attributed to the stars alone. The consensus is the extra emission is due to the accretion of the Inter-Stellar Medium (ISM) onto the central Super-Massive Black Hole (SMBH). It has become increasingly clear in recent years that there exists to distinct modes of AGN: radiative and jet modes. This are explored in the following sections.

	\subsection{Radiative Mode}
		\label{subsec:introRadiative}

	\subsection{Jet Mode}
		\label{subsec:introJet}
