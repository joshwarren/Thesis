\begin{abstractlong}
There is a strong consensus within the scientific literature that the most powerful radio galaxies are the result of gas-rich, major mergers causing large amounts of material to fall onto the black hole at the centre of the resulting galaxy, thus producing an immensely powerful radio jet. However, such galaxies are by construction extremely rare and are not representative of the general radio galaxy population. Most radio galaxies have low power, and their basic properties and origins are poorly constrained. One of the most popular suggestions is that they are bona-fide detections of jet-mode active galactic nuclei (AGN), a class of AGN thought to maintain the dearth of star formation in early-type galaxies (ETGs), after most of the fuel for star formation has been removed by an earlier event. If true, then all ETGs must regularly go through such an active phase, and low-powered radio galaxies should be much the same as their radio-quiet (quiescent phase) ETG counterparts. 

To test this hypothesis, a complete sample of 11 nearby low-powered radio galaxies was constructed and observed with a suite of instruments. We focus here on the spatially-resolved stellar kinematics, stellar populations, and ionized gas distribution, kinematics and ionization derived from integral-field spectroscopic (IFS) observations with the Visible Multi-object Spectrograph (VIMOS) and archival data from the Multi-unit Spectroscopic Explorer (MUSE), both on the Very Large Telescope (VLT). We compare the characteristics and properties (accessible with $V$-band IFS data) of our sample of low-power radio galaxies to those of ordinary (i.e.\ optically-selected) ETGs from the Atlas$^\text{3D}$ and MASSIVE IFS surveys.

For the stellar component of our sample galaxies, we use the kinematic maps to identify regular/non-regular rotators and kinematic substructures. In our sample 4 galaxies are regular rotators and two (with a tentative third) galaxies were found to contain kinematically decoupled cores (KDCs). We use the $\lambda_R$ specific angular momentum parameter from Atlas$^\text{3D}$ to classify the galaxies according to the fast/slow rotator classification scheme. We found a similar fraction ($45\pm13$\%) of slow rotators to the combined Atlas$^\text{3D}$ and MASSIVE samples, corrected for mass distribution. We also find the fast rotators tend to have aligned kinematic and photometric position angles, while the slow rotators can have any angle between the two position angles. The best-fitting stellar populations show our sample to be typical massive ETGs: i.e.\ mostly old, metal rich and alpha-element enhanced. We found and measure the radial stellar population gradients and found averages of $\Delta_\text{log age} = 0.014\pm0.007 \,\mathrm{dex \, arcsec^{-1}}$ and $\Delta_\text{[Fe/H]} = -0.03\pm0.01 \, \mathrm{dex \, arcsec^{-1}}$. The KDCs found previously were observed to be large and contain old stellar populations and are the result of galaxy mergers. 

Some of the literature suggestion that if cold gas is the fuel for the AGN then unless it is accreting in a highly turbulent way, it might be expected to form stars. However we see no evidence of a very young stellar population right at the centre of our sample galaxies. 

For the ionized gas components of our sample galaxies, the estimated gas masses were found to be in the range of $10^4$\,--\, $10^6\,\mathrm{M_\odot}$, which is the upper end of the range expected from optically-selected ETGs. This was the only suggestion we find that our sample galaxies may be intrinsically different than optically-selected ETGs. We detected spatially extended gas in 4 of our sample, with a range of kinematics. Two galaxies had ionized gas kinematics misaligned to that of the stars, suggesting and external origin to the gas, while the kinematics of one was aligned with that of its stars (consistent with an internal origin). The final galaxy was unsettled and thus not clear. 

Using the classic Baldwin\,--\,Phillips\,--\,Terlevich (BPT) diagnostic plots, as well as similar alternatives, for galaxies with observations outside of the wavelength range required for the BPT plots, we find the dominant ionization source for each galaxy. We find that 9 of our sample galaxies are low-ionized nuclear emission region galaxies (LINERs; 5 of which can be attributed to the radiation field from the AGN), 1 is a Seyfert II and 1 a passive (line-less) galaxy. 

All of the properties found are consistent with that of optically selected ETGs with a similar mass distribution. We thus find no conclusive difference between the stellar and ionized gas properties all of these ETGs except for a tentative suggestion of slightly higher ionized gas masses in our radio galaxies. If confirmed, the latter may be gas built up during the quiescent phase of ETGs, ultimately triggering the active (radio) phase itself then re-removing such gas from the galaxy.

Some of the sample galaxies have clear evidence of merger events (e.g.\ large KDCs), but others have a complete lack of evidence of a merger, suggesting that radio galaxies can form without the requirement of a merger event. Indeed, it is well known that NGC 612 harbours a large extended gas disc with kinematics that are consistent with an internal origin of the gas. As such, in rare cases, it is possible that radio galaxies can be fueled through entirely secular processes.
\end{abstractlong}