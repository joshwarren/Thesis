\begin{abstractlong}
There is a strong consensus within the scientific literature that the most powerful radio galaxies are the result of gas-rich, major mergers causing large amounts of material to fall onto the black hole at the centre of the resulting galaxy, producing an immensely powerful radio jet. However such galaxies are extremely rare and are not representative of the general population of radio galaxies. So what of the low-powered radio galaxies? 

One of the most popular suggestions is that they are bona-fide detections of jet-mode active galactic nuclei (AGN), a class of AGN thought to be responsible for maintaining the lack of star formation within early-type galaxies (ETGs) after most of the fuel available to form stars has been removed by an earlier event. If true, all ETGs must regularly go through such active phases and as such, in order to maintain their low star-formation rates, low-powered radio galaxies should appear much the same as their radio-quiet (quiescent phase) ETGs counterparts. 

In order to test this hypothesis, a complete sample of 11 nearby low-powered radio galaxies was constructed. This sample has been subjected to a suite of observations by our group, using the sub-millimeter Atacama Pathfinder Experiment (APEX) and Atacama Large Millimeter/sub-millimeter Array (ALMA) to observe the spatially resolved kinematics of the molecular gas; and the Visible Multi-object Spectrograph (VIMOS) on the Very Large Telescope (VLT) in integral-field spectroscopy (IFS) mode to observe the spatially resolved kinematics of both the stars and the ionized gas. In this thesis we present the reduction and analysis of the VIMOS observations together with additional archival Multi-unit Spectroscopic Explorer (MUSE) observations of 4 of the sample galaxies. We compare the characteristics and properties (accessable with $V$-band IFS data) of ordinary (i.e.\ optically-selected) ETGs from the Atlas$^\text{3D}$ and MASSIVE IFS surveys to our sample of radio galaxies.

We find no conclusive differences in the stellar and ionized gas properties between our sample of low-powered radio galaxies and the optically selected ETG galaxies of the Atlas$^\text{3D}$ and MASSIVE surveys, except for a tentative suggestion of slightly higher masses of ionized gas in our radio galaxies. We suggest, if real, that this may be the effect of gas built up during the quiescent phase of ETGs which ultimately prompts the active (radio) phase, re-removing such gas from the galaxy.

The existance of radio galaxies with aligned stellar and ionized gas kinematics coupled with the fact that disc radio galaxies are extremely rare, suggests that while the formation of jet-mode AGN does not explicitly require a dry major merger (which destroys the disc, resulting in a slow rotator, and expels internal gas reservoirs), such an event appears to significantly enhances the likelihood of it occurring. 
\end{abstractlong}