\begin{abstractlong}
There is a strong consensus within the scientific literature that the most powerful radio galaxies are the result of gas-rich, major mergers causing large amounts of material to fall onto the black hole at the centre of the resulting galaxy, thus producing an immensely powerful radio jet. However, such galaxies are by construction extremely rare and are not representative of the general radio galaxy population. Most radio gaalxies have low power, and their basic properties and originas are poorly constrained. One of the most popular suggestions is that they are bona-fide detections of jet-mode active galactic nuclei (AGN), a class of AGN thought to maintain the dearth of star formation in early-type galaxies (ETGs), after most of the fuel for star formation has been removed by an earlier event. If true, then all ETGs must regularly go through such an active phase, and low-powered radio galaxies should be much the same as their radio-quiet (quiescent phase) ETG counterparts. 

To test this hypothesis, a complete sample of 11 nearby low-powered radio galaxies was constructed and observed with a suite of instruments. We focus here on the spatially-resolved stellar kinematics, stellar populations, and ionized gas distrabution, kinematics and ionization derived from integral-field sepectric (IFS) observations with the Visible Multi-object Spectrograph (VIMOS) and archival data from the Multi-unit Spectroscopic Explorer (MUSE), both on the Very Large Telescope (VLT). We compare the characteristics and properties (accessible with $V$-band IFS data) of our sample of low-power radio galaxies to those of ordinary (i.e.\ optically-selected) ETGs from the Atlas$^\text{3D}$ and MASSIVE IFS surveys.

We find no conclusive difference between the stellar and ionized gas properties all of these ETGs except for a tentative suggestion of slightly higher ionized gas masses in our radio galaxies. If confirmed, the latter may be gas built up during the quiescent phase of ETGs, ultimately triggering the active (radio) phase itself then re-removing such gas from the galaxy.

The existence of radio galaxies with aligned stellar and ionized gas kinematics, coupled with the fact that radio-loud disc galaxies are extremely rare, suggests that while the formation of jet-mode AGN does not explicitly require a dry major merger (that destroys the disc, resulting in a slow rotator, and expels internal gas), such an event appears to significantly enhances its likelihood. 
\end{abstractlong}

