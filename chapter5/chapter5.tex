\chapter{Ionized Gas Distribution, Kinematics and Ionization}
	\label{cha:gas}
The interstellar medium (ISM) of early-type galaxies (ETGs) has several components: a diffuse hot ($\sim 10^7 \, \mathrm{K}$) X-ray halo (with typical mass $10^8$--$10^{10} \, \mathrm{M_\odot}$); a warm ($\sim 10^4 \, \mathrm{K}$) ionized gas component ($10^2$--$10^5 \, \mathrm{M_\odot}$), which can be more clumpy; and cold ($<10^2 \, \mathrm{K}$) atomic and molecular gas ($10^6$--$10^8 \, \mathrm{M_\odot}$), which is generally confined to small (kpc or less) clouds. In this chapter we study the spatially-resolved properties of the ionized gas component of radio galaxies (RGs), exploiting the emission lines in the VIMOS and MUSE datacubes of the Southern Sample.

Using the methods described in Section \ref{subsec:EmissionFit}, we find the best-fit line-of-sight velocity distribution (LOSVD; assumed to be Gaussian and parametrised by the mean velocity and velocity dispersion only) of the emission lines in each bin. As described in \ref{subsec:EmissionFit}, all emission lines are fit with the same LOSVD, but each with its own flux. In the cases of the [\ion{O}{iii}], [\ion{O}{i}] and [\ion{N}{ii}] doublets, the two components of the doublets are fit with a fixed flux ratio of 1:0.34. The [\ion{N}{i}] doublet has a fixed ratio of 1:0.65 \citep{Safier1992}. The two components of the [\ion{O}{ii}] and [\ion{S}{ii}] doublets are fit independently.

This chapter is structured as follows. Firstly in Section \ref{sec:gasFlux} the flux and equivalent-width maps of each emission line in the respective wavelength range of VIMOS and MUSE are shown. Gas masses are calculated and upper limits estimated in the cases of non-detections (see Section \ref{sec:GasFlux}). Secondly, in Section \ref{sec:GasKin} the kinematics of the ionized gas is discussed. Thirdly, the likely sources of the gas ionization are investigated, making use of several emission line diagnostics (Section \ref{sec:Diagnostics}). The results for each galaxy in the Southern Sample are summarised in Table \ref{tab:gasMass}. Finally, we conclude in Section \ref{sec:gasDiscussion} with a discussion of the results of this chapter.


\begin{table}
	\centering
\begin{threeparttable}
	\caption{Gas masses of the Southern Sample galaxies.}
	\label{tab:gasMass}
	% \begin{tabular*}{\textwidth}{@{\extracolsep{\fill}}l r r r l}
	\begin{tabular}{l c c c c}
		\hline
		\hline
		Galaxy & \multicolumn{2}{c}{\ion{H}{ii} Mass} & Balmer & LINER/ \\
		& VIMOS\tnote{a} & MUSE & Decrement & Seyfert \\
		& ($\log\mathrm{M_\odot}$) & ($\log\mathrm{M_\odot}$) & \\
		\hline
		ESO 443-G024 & $5.31 \pm 0.01$ 	& --  		& -- & No detections \\
		IC 1459 	& $5.25 \pm 0.01$	& $5.51 \pm 0.01$ & $4.57 \pm 0.06$ & LINER-AGN\\
		IC 1531 	& $5.04 \pm 0.01$	& -- 		& -- & Seyfert 2\\
		IC 4296		& $5.43 \pm 0.01$	& $< 4.14$ 	& $<10.9$\tnote{b} & LINER-AGN \\
		NGC 612 	& $6.21 \pm 0.01$ 	& -- 		& -- & LINER-AGN \\
		NGC 1316 	& -- 				& $ 5.29 \pm 0.01$ & $3.54 \pm 0.05$ & LINER-AGN \\
		NGC 1399 	& $3.94 \pm 0.02$ 	& $ 4.54 \pm 0.01$ & \tnote{c} & LINER \\
		NGC 3100 	& $5.27 \pm 0.01$	& -- 		& -- & Seyfert 2 \\
		NGC 3557 	& $4.89 \pm 0.01$ 	& -- 		& -- & LINER/Retired \\
		NGC 7075 	& $5.36 \pm 0.01$	& -- 		& -- & LINER \\
		PKS 718-34  & $< 5.08$	 		& -- 		& -- & No detections \\
		\hline
		\hline
	\end{tabular}
	\begin{tablenotes}
	\footnotesize
	\note Col.\,1: Galaxy. Col.\,2: \ion{H}{ii} mass derived from the H$\beta$ line in the VIMOS data, assuming a Balmer decrement of 2.85. Col.\,3: \ion{H}{ii} mass derived from the H$\alpha$ line in the MUSE data. Col.\,4: Balmer decrement measured from the MUSE data. Col.\,5: Main source of ionizing radiation (see Section \ref{sec:Diagnostics} for a description of the different classes). A -- means we have no data.
	\item [a] The VIMOS flux calibration and thus the derived gas masses are only approximate (see Section \ref{subsec:VIMOSreduction} for more detail on the flux calibration).
	\item [b] H$\beta$ was not detected with A/N $> 2.5$, so the fit is unreliable (hence an upper limit). 
	\item [c] H$\beta$ was not detected, so the formal measured Balmer decrement is infinite. 
	\end{tablenotes}
\end{threeparttable}
\end{table}


\section{Ionized Gas Distribution}
	\label{sec:GasFlux}

	\subsection{Maps}
		\label{subsec:GasMaps}
		Images of the Southern Sample galaxies in the [\ion{O}{iii}]$\lambda\lambda$4957,5007 lines are shown in Figs.\ \ref{fig:VIMOS_OIII} and \ref{fig:MUSE_OIII}. Only 4 galaxies of the Southern Sample have emission lines detected outside of their central region (IC 1459, NGC 612, NGC 1316 and NGC 3100; see Figs.\,\ref{fig:VIMOS_OIII} and \ref{fig:MUSE_OIII}). Of the remaining 7 galaxies, NGC 1399 and PKS 718-34 have no detection of H$\beta$ in their spatially-resolved map (although we do detect H$\beta$ in the spatially-integrated spectrum of NGC 1399; see Section \ref{subsec:GasMass}), while the other 5 galaxies are only detected in their centre. All other emission lines have similar distributions, but with different fluxes.%, with the exception of NGC 612, where the significant cloud in [\ion{O}{iii}] to the south of the centre of the galaxy is not seen in H$\beta$ (see Fig.\,\ref{fig:NGC612_Hb}). 


		\begin{figure}
			\centering
			\includegraphics[width=\textwidth]{chapter5/vimos/Hb.png}
			\caption[VIMOS \bracket{\ion{O}{iii}} maps]{VIMOS [\ion{O}{iii}]$\lambda\lambda$4957,5007 maps. Total flux contours (isophotes) are shown in black, \ce{^{12}CO(2-1)} contours from ALMA in cyan, and radio continuum contours from VLA in green. The radio band shown depends on the spatial resolution and extent of the datasets available, selected to best match our IFS data.\label{fig:VIMOS_OIII}} 
			
		% \end{figure}
		% \begin{figure}
		% 	\centering
			\vspace{\floatsep}
			\includegraphics[width=\textwidth]{chapter5/muse/Hb.png}
			\caption[MUSE \bracket{\ion{O}{iii}} maps]{MUSE [\ion{O}{iii}]$\lambda\lambda$4957,5007 maps.\label{fig:MUSE_OIII}} 
			
		\end{figure}


		

		% \begin{figure}
		% 	\centering
		% 	\includegraphics[width=0.4\textwidth]{chapter5/vimos/ngc0612_Hb.png}
		% 	\caption[NGC 612 H$\beta$ image]{NGC 612 H$\beta$ map showing a large cloud to the south of the centre of the galaxy, which is not present in the H$\beta$ map (see Fig.\ref{fig:VIMOS_Hb}).} 
		% 	\label{fig:NGC612_Hb}
		% \end{figure}


		% Because equivalent width is not dependent on overall flux (except for the initial detection threshold: a line must be detected in order to observe its equivalent width), it can be used to highlight local structures that cannot be seen in flux maps. The equivalent width maps for each emission line for each galaxy in the Southern Sample are shown in Figs.\,\ref{fig:VIMOS_ew} and \ref{fig:MUSE_ew}. 

		% \begin{figure}
		% 	\centering
		% 	\includegraphics[width=\textwidth]{chapter5/vimos/ew.png}
		% 	\caption[VIMOS equivalent width maps]{VIMOS equivalent width maps.} 
		% 	\label{fig:VIMOS_ew}
		% \end{figure}
		% \begin{figure}
		% 	\centering
		% 	\includegraphics[width=\textwidth]{chapter5/muse/ew.png}
		% 	\caption[MUSE equivalent width maps]{MUSE equivalent width maps.} 
		% 	\label{fig:MUSE_ew}
		% \end{figure}

	\subsection{Gas Masses}
		\label{subsec:GasMass}

		Total ionized gas masses are derived using the prescription of \citet{Sarzi2005}. This is a very rough calculation and the values should not be used for quantitative applications. The approach follows \citet{Kim1989}, whereby
		\begin{equation}
			\left(\frac{M_\text{\ion{H}{ii}}}{M_\odot}\right) = 280 \left(\frac{D}{10\, \mathrm{Mpc}}\right)^2 \left(\frac{F(\mathrm{H\alpha})}{10^{-14} \, \mathrm{erg \, s^{-1} \, cm^{-2}}}\right) \left(\frac{10^3 \, \mathrm{cm^{-3}}}{n_\mathrm{e}}\right) \, ,
		\end{equation}
		where $M_\text{\ion{H}{ii}}$ is the total mass of \ion{H}{ii} in the galaxy, $D$ is the galaxy distance, and $F(\mathrm{H\alpha})$ is the total observed galaxy H$\alpha$ flux. This method assumes an electron density $n_\mathrm{e} = 100 \, \mathrm{cm^{-3}}$, a temperature of $10^4$ K and case-B recombination (electrons above 13.6\,eV are not reabsorbed; e.g.\ \citealt[p.\,74]{Osterbrock1974}). Like \citet{Sarzi2005}, we only claim a detection if the amplitude-to-noise ratio (A/N) is $>4$ for [\ion{O}{iii}] and $>2.5$ for H$\beta$ or H$\alpha$. Since most of the gas is centrally concentrated, a large field of view, often only serves to increase the noise (thus reducing A/N). 
		% Our lower limits on the gas mass, are set by the mass measured from the largest aperture that still meets our detection criteria. This would be the gas mass of the galaxy if there was no gas at larger radii. 
		Upper limits are calculated using the $1\sigma$ noise level (see Section \ref{subsec:EmissionFit}) or the H$\alpha$ (H$\beta$, in the case of VIMOS data) flux measurement that does not meet out A/N criterion, whichever is larger, assuming that the gas is distributed over the entire field of view.


		
		

% So what?




\section{Ionized Gas Kinematics}
	\label{sec:GasKin}
	In Figs.\,\ref{fig:VIMOS_Gaskine} and \ref{fig:MUSE_Gaskine}, we show maps of the mean velocity and velocity dispersion of the ionized gas for the 4 galaxies with extended emission lines, IC 1459, NGC 612, NGC 1316 and NGC 3100. Due to that fact that, unlike stars, gas is dissipational, it is to be expected th at given sufficient time, the ISM will always settle into a disc. These maps show that the ISM of the Southern sample galaxies varies from largely settled discs in IC 1459 and NGC 612 to more disordered kinematics of NGC 3100 and a possible inflow in NGC 1316.

	\begin{figure}
		\centering
		\includegraphics[height=0.47\textheight]{chapter5/vimos/kin.png}
		\caption[VIMOS ISM kinematic maps]{VIMOS ISM kinematic maps. Left to right: mean velocity and velocity dispersion maps. Alternate columns show a given parameter and its associated uncertainty. Top to bottom: IC 1459, NGC 612 and NGC 3100. Contours are as in Fig.\ \ref{fig:VIMOS_OIII}. Colour scale ranges are: mean velocity maps -360 to 360\,$\mathrm{km \, s^{-1}}$ (except for NGC 3100 which has a range of -100 to 100\,$\mathrm{km \, s^{-1}}$), mean velocity uncertainty 1 to 15\,$\mathrm{km \, s^{-1}}$, velocity dispersion 35 to 200\,$\mathrm{km \, s^{-1}}$, and velocity dispersion uncertainty 1 to 20\,$\mathrm{km \, s^{-1}}$.} 
		\label{fig:VIMOS_Gaskine}
	\end{figure}

	\begin{figure}
		\centering
		\includegraphics[height=0.31\textheight]{chapter5/muse/kin.png}
		\caption[MUSE ISM kinematic maps]{As Fig.\ \ref{fig:VIMOS_Gaskine} but for the MUSE ISM kinematics.}
		\label{fig:MUSE_Gaskine}
	\end{figure}


% Separate section?
	% \subsection{Misalignment of Gas}
	% 	\label{subsec:Gasmisaligment}
	It is immediately clear when comparing the mean stellar velocity maps (Figs.\,\ref{fig:VIMOS_stellar} and \ref{fig:MUSE_stellar}) to the mean gas velocity maps (Figs.\,\ref{fig:VIMOS_Gaskine} and \ref{fig:MUSE_Gaskine}) that where gas is detected, its angular momentum is not often aligned with that of the stars. A misalignment between the kinematics of the stars and gas suggests an external origin for the gas (e.g.\ accretion or wet merger). However it should be noted that the conversus is not necessarily true: aligned kinematics is consistent with an internal origin (e.g.\ stellar-mass loss), but it does \emph{not} require it \citep[e.g.][]{Davis2011a}. 

	We now comment on each galaxy in turn.

	\paragraph{IC 1459} has ionized gas counter-rotating with respect to the stellar kinematically-decoupled core, and also rotating independently from the rest of the galaxy. The gas is rotating with a mean velocity an order of magnitude higher than that of the stars (outside of the KDC). As the stellar disc did not survive the merger even that created the KDC, it is unlikely that the gas in IC 1459 progenitor galaxy survived. The lost gas may, however, have been re-accreted. Alternatively, the ISM may have yet another external origin. 

	\paragraph{NGC 612} has well aligned stellar and gaseous kinematics. The misalignment between the position angle of the stellar and gas kinematics (using the method described in Section \ref{subsec:Misalignment}) is just $1\fdg0\pm0\fdg7$, consistent with an internal origin for the gas. This adds to the evidence (see Section \ref{sec:NGC612}) that NGC 612 has not undergone a major merger (major mergers exhaust galaxies of their gas and destroy a stellar disc; see Section \ref{sec:stellarDiscussion}) and suggests that, the fueling and powering of the radio jet must be a purely secular process. 


	\paragraph{NGC 1316} is a complex object. The gas disc is consistent with being aligned with the stellar kinematics, but with a large inflow towards the nucleus from the south-west (see Fig.\,\ref{fig:Inflow}), almost perpendicular to the radio jet. However, other interpretations are possible, and better quality data (longer exposure times) are required to add certainty to this interpretation. Nevertheless, it is clear that the gas in not in a settled disc. 

	\begin{figure}
		\centering
		\includegraphics[width=\textwidth]{chapter5/ngc1316_inflow.png}
		\caption[Inflows in NGC 1316]{Ionized gas inflows in NGC 1316.} 
		\label{fig:Inflow}
	\end{figure}


	\paragraph{NGC 3100} has an interesting ISM morphology. The \ce{^{12}CO(2-1)} contours (shown in cyan in all figures) show a broken ring with the radio jet passing through the gaps in the ring (Ruffa et\,al., in prep.). We observe the ionized gas to be brightest in the gaps of this ring. The simplest comment that we can make of the ionized gas kinematics is that it is completely different than that of the stars. Secondly, that the centre of rotation of the gas appears to be offset from the centre of the galaxy (as defined by the stellar surface brightness) by several arcseconds to the west. 

	\subsection{Kinematic mis-alignments}

		\citet{Davis2011a} showed that $36\pm5$\% of fast rotators have ionized gas kinematics misaligned with respect to the stellar kinematics, while slow rotators have a flat distribution of misalignments. This indicates that slow rotators are dominated by external sources of gas, while fast rotators can have either internal or external sources. This is consistent with our observations of the Southern Sample: the only slow rotator with detected ionized gas is IC 1459, which has significantly kinematically-misaligned gas; of the 3 fast rotators, only 1 (NGC 3100) is definitely misaligned, while the other 2 (NGC 612 and NGC 1316) are consistent with an internal gas origin.



\section{Ionization Sources}
	\label{sec:Diagnostics}
	Determining the sources of ionizing radiation in galaxies using emission line ratios, such as Baldwin--Phillips--Terlevich (BPT) plots (\citealt{Baldwin1981}; revised by \citealt{Kewley2001, Kewley2006} and \citealt{Kauffmann2003a}), has become increasingly widespread exercise. That said, there are several important caveats. Firstly, great care must be taken when applying each diagnostic; much of the literature misinterprets the resulting classifications. Secondly, in the absence of spatially-resolved spectra, low-ionization nuclear emission-line region (LINER) classifications have often been taken as a marker for jet-mode active galactic nuclei (AGN). As discussed in Section \ref{subsubsec:JetFeedback} however, several recent surveys have shown that many of these may not be bona fide AGN \citep[e.g.][]{Sarzi2005, Sarzi2010, Singh2013, Belfiore2016a}. In these cases, the emission may not even originate exclusively from the centres of the galaxies, the location of the putative AGN. This led to the creation of the low-ionization emission-line region (LIER) classification, with the same criteria as LINERs, but not necessarily restricted to the nuclear regions.

	The problem with classifying ETGs by their emission lines is that they often have very little ionized gas and weak ionizing radiation fields, so that it is often difficult to detect emission lines from their ISM. Furthermore, the BPT plots require a larger spectral range than many integral-field spectrographs (IFS) deliver. This has given rise to a number of other, analogous, classifying plots, that we take advantage of here. A description of our ionization source classifying process follows below. 

	Firstly, we preferentially use the BPT plots for the galaxies observed with MUSE (that has the required spectral range; see Section \ref{subsec:BPT}). This allows us to classify galaxies into star-forming, LINER/LIER or Seyfert 2 classes. For the emission-line fluxes derived from the VIMOS datacubes, we use the [\ion{N}{i}]/H$\beta$ versus [\ion{O}{iii}]/H$\beta$ plot of \citet[hereafter the SAURON plot]{Sarzi2010}. Unlike the BPT plots, \citet{Sarzi2010} do not define classification boundaries. We use the Balmer decrement (see Section \ref{subsec:Ndec}) and the [\ion{N}{i}] flux versus [\ion{N}{ii}] flux relation (Section \ref{subsec:Ndec}) to convert the [\ion{N}{ii}]/H$\alpha$ BPT plot classification boundaries of \citet{Kewley2001} and \citet{Kauffmann2003a} to the [\ion{N}{i}]/H$\beta$ SAURON plot (as described in more detail in Section \ref{subsec:Ndec}). This allows us to classify galaxies into star-forming or LINER/Seyfert 2 classes. We also use the so-called WHaN2 plot (H$\alpha$ equivalent width versus [\ion{N}{ii}]/H$\alpha$; see Section \ref{subsec:WHaN2}) from \citet{CidFernandes2011}. Using the same method as above, we again transform this plot into a H$\beta$ equivalent width versus [\ion{N}{i}]/H$\beta$ (WHbN1) plot for the VIMOS data. To transform the equivalent widths of H$\alpha$ to H$\beta$, we find the continuum at H$\alpha$, $C_\text{6563\AA}$, and plot it against the continuum at H$\beta$, $C_\text{4861\AA}$ (again described in more detail in Section \ref{subsec:Ndec}). This allows us to classify galaxies into star-forming, strong AGN (Seyfert 2), weak AGN (LINER) or retired classes. 

	To check if the LINER classifications are due to a central radiation source (such as an AGN) or extended sources (such as weak star formation or post-asymptotic giant branch stars; pAGB stars), we examine the H$\alpha$ and H$\beta$ flux radial profiles of our galaxies with extended detected ionized gas (Section \ref{subsec:Hb}). Finally, in Section \ref{subsec:MEx} we use the mass\,--\,excitation (MEx) plot of \cite{Nyland2016}, using parameters derived from the central region of the galaxies, only measured within a 3$^{\prime\prime}$ wide aperture. This allows us to classify into star-forming, Seyfert 2, LINER, transition or passive classes, with LINER galaxies subdivided into those whose LINER behaviour is (LINER-AGN) or is not attributed to a central AGN. 

	% \subsection{Using Emission Line Beyond the VIMOS Wavelength Range}
	\subsection{Extrapolating VIMOS Data}
		\label{subsec:Ndec}

		The Balmer decrement, $d_\mathrm{H}$, is the ratio of the H$\alpha$ to H$\beta$ fluxes. There are good theoretical motivations for this ratio to be near constant at $d_\mathrm{H} = 2.86$, under a wide variety of conditions. In reality, however, higher values are routinely observed. These can be explained as due to either the presence of dust causing extinction (whereby emission lines at shorter wavelengths appear dimmer than expected from their redder counterparts due to dust preferentially scattering and absorbing bluer light) or some mechanism that populates hydrogen levels from the ground up. Such mechanisms may be important in high-density environments, such as within powerful (type 1) AGN \citep[e.g.][]{Shields1974, Netzer1975}. 
		%
		The majority ($\approx 60$\%) of ETGs and all LINER and Seyfert galaxies are dusty \citep[e.g.][]{Martini2013}. 
		Such galaxies reveal clear dust lanes in \textit{Hubble Space Telescope (HST)} observations \citep[e.g.][]{Martini2013}, are bright in the infrared (where the thermal emission from dust is seen as a dust ``bump''; e.g.\ \citealt{Jura1987, Knapp1992}), and possess other emission lines \citep[e.g.\ \bracket{\ion{S}{ii}};][]{Wampler1968} showing similar reddening. Thus, assuming the observed steepening of the Balmer decrement is entirely due to dust, the ratio of the observed decrement to the theoretical value of 2.86 is a good estimate of the dust reddening.

		If we assume that different ionization states of a given atom occur in the same spatial location, then shielding is not important or at least is not non-linear in its effect on the emission-line ratios. Given that the different species are then produced under the same conditions, it is natural to expect a simple relationship between them. Using the emission line fluxes derived from our MUSE data, we find that the relationship between [\ion{N}{ii}] flux and [\ion{N}{i}] flux is approximately linear (see Fig.\,\ref{fig:NII_NI}). We emphasize that this is a purely empirical correlation and thus differs substantially from the Balmer decrement with its strong theoretical basis. 

% I actually work with luminosity - how will that effect things?
		To find the intrinsic conversion from [\ion{N}{i}] flux to [\ion{N}{ii}] flux we must first correct for dust reddening. Assuming that the extinction can be approximated by a linear form in the $V$-band, we correct the flux of the bluer emission line of the pair $F_\mathrm{b}$, using
		\begin{equation}
			F^\mathrm{corr}_\mathrm{b} = F_\mathrm{b} \left(\frac{\Delta\lambda}{\lambda_\mathrm{H\alpha} - \lambda_\mathrm{H\beta}}\right) \left(\frac{F_\mathrm{H\alpha}}{2.86 F_\mathrm{H\beta}}\right) \, ,
		\end{equation}
		where $\Delta\lambda$ is the difference in wavelength between the two lines, $\lambda_\mathrm{H\alpha}$ and $\lambda_\mathrm{H\beta}$ are the rest-frame wavelengths of the H$\alpha$ (6563 \AA) and H$\beta$ (4861 \AA) emission lines with fluxes of $F_\mathrm{H\alpha}$ and $F_\mathrm{H\beta}$, respectively and $F^\mathrm{corr}_\mathrm{b}$ is the dust-corrected flux of the bluer line. The intrinsic conversion can then be found by comparing $F^\mathrm{corr}_\mathrm{b}$ to $F_\mathrm{r}$, the flux of the redder line in the pair. 

		In Fig.\,\ref{fig:NII_NI}, we fit a straight line to the [\ion{N}{ii}] flux versus [\ion{N}{i}] (flux corrected for dust extinction) relation of IC 1459 and NGC 1316 using the least-trimmed squares routine, \textsc{lts\_linefit} of \citet{Cappellari2013} due to its robust handling of the uncertainties along both axes. 
% THIS NEEDS SOME WORK!!!  *****^^^^^^^^^
		As can be seen in the figure, there two galaxies have quite different gradients. We require a single function in order to be able to use [\ion{N}{i}] as a proxy for [\ion{N}{ii}] for our VIMOS data. We thus fit each galaxy independently and use the mean gradient, neglecting the intercept. We find a mean gradient of $12.23 \pm 0.14$.

		\begin{figure}
			\centering
			\includegraphics[width=0.7\textwidth]{chapter5/NII_NI_ratio.png}
			\caption[The nitrogen `decrement']{Nitrogen `decrement': plot of the [\ion{N}{ii}] versus [\ion{N}{i}] flux for the galaxies IC 1459 and NGC 1316, used to transform the classification boundaries of line ratio diagnostics involving [\ion{N}{ii}] to those involving [\ion{N}{i}]. The line fluxes have been corrected for the galaxies' distances.} 
			\label{fig:NII_NI}
		\end{figure}

		% Some of the brightest bins in Figs.\,\ref{fig:NII_NI} and \ref{figLOI_OIII} deviate away from the best-fitting line, with an excess in the bluer line. Given that these bins are located at the centre of the galaxy, we suggest that we are possibly over correcting for dust in these regions and that some of the reddening for the Balmer decrement is actually due to the different conditions close to the AGN. 

		To use the H$\beta$ equivalent width instead of the H$\alpha$ equivalent width, we require not only the Balmer decrement, but also the stellar `decrement', i.e.\ the ratio of the continuum level at H$\alpha$ to that at H$\beta$. In the same manner as we did for the Nitrogen `decrement', we find in Fig.\,\ref{fig:stellarDec} an approximately linear relationship between the continuum level at 6563\,\AA, $C_\text{6563\AA}$, and the continuum level corrected for dust extinction at 4861\,\AA, $C^\text{corr}_\text{4861\AA}$, using our MUSE data of IC 1459 and NGC 1316. Using the data shown in Fig.\,\ref{fig:stellarDec}, we find
		\begin{equation}
			C_\text{6563\AA} = \,  (0.142\pm0.004757) \times 10^{-6} + (1.047\pm0.008) C^\text{corr}_\text{4861\AA} \, .
		\end{equation}
		This is used to be transform the classification boundaries of diagnostics dependent on the equivalent width of H$\alpha$ to those involving H$\beta$. 

		\begin{figure}
			\centering
			\includegraphics[width=0.7\textwidth]{chapter5/stellar_ratio.png}
			\caption[The stellar `decrement']{Stellar `decrement'. Plot of the continuum flux at H$\alpha$ versus the continuum flux at H$\beta$ for the galaxies IC 1459 and NGC 1316, used to transform the classification boundaries of diagnostics involving the H$\alpha$ equivalent width to those involving H$\beta$. As in Fig.\,\ref{fig:NII_NI}m the fluxes have been corrected for the galaxies' distances.} 
			\label{fig:stellarDec}
		\end{figure}

	\subsection{BPT Diagnostics}
		\label{subsec:BPT}
		Firstly, in Fig.\,\ref{fig:BPT}, we examine our galaxies using the classic BPT diagnostic plots. For the plots using [\ion{N}{ii}]/H$\alpha$, [\ion{S}{ii}]/H$\alpha$ and [\ion{O}{i}]/H$\alpha$ versus [\ion{O}{iii}]/H$\beta$, we can only use the emission-line fluxes derived from our MUSE datacubes, as the lines necessary for gauging the hardness of the ionizing radiation field are outside the spectral range of VIMOS. Only two MUSE galaxies have detections of the necessary emission lines: IC 1459 and NGC 1316. Both occupy very similar positions in the BPT plots, on or close to the boundary between the Seyfert 2 and LINER classes. On the whole, the central bins tend to be on the LINER side.

		\begin{figure}
			\centering
			\includegraphics[width=\textwidth]{chapter5/BPT.png}
			\caption[BPT plots]{BPT plots for IC 1459 and NGC 1316. The colour scale (blue to yellow) represents increasing distance from the galaxy centre. The classification boundaries are from \citet{Kewley2006}.}
			\label{fig:BPT}
		\end{figure}

		\begin{figure}
			\centering
			\includegraphics[width=0.8\textwidth]{chapter5/SAURON.png}
			\caption[An alternative diagnostic plot]{SAURON diagnostic plot for the line ratios available in the VIMOS datacubes. To show the expected position of LINERs on the SAURON plot, the MAPPINGS-III shock model grid of \citet{Allen2008} is shown in green. The solid lines are lines of constant shock velocity (from 150 to 1000\,$\mathrm{km\,s^{-1}}$) and the dotted lines are lines of constant magnetic parameter $b \equiv B/\sqrt{n}$ (from 0.5 to 4.0), where $B$ is the magnetic field strength and $N$ the (pre-shock) particle number density \citep{Dopita1996}. All models assume an electron density of $n_\mathrm{e} = 1\,\mathrm{cm^{-3}}$.}
			\label{fig:SAURON}
		\end{figure}

		\citet{Sarzi2010} showed that Seyferts, LINERs and star-forming galaxies are reasonable separated in the [\ion{N}{i}]/H$\beta$ versus [\ion{O}{iii}]/H$\beta$ diagnostic plot (hereafter the SAURON diagnostic plot). To exploit the VIMOS datacubes we thus use the [\ion{N}{ii}] -- [\ion{N}{i}] relation derived in Section \ref{subsec:Ndec}, along with the Balmer decrement, to transform the \citet{Kewley2001} extreme starburst and the \citet{Kauffmann2003a} pure star-formation boundaries in the [\ion{N}{ii}]/H$\alpha$ versus [\ion{O}{iii}]/H$\beta$ plot to boundaries in the SAURON plot. For the MUSE datacubes, only IC 1459 and NGC 1316 have detections of [\ion{N}{i}]. Given that we already have the superior classic BPT plots for these galaxies, we only show the VIMOS-derived SAURON diagnostic plot. This is shown in Fig.\,\ref{fig:SAURON} for all 6 VIMOS galaxies with detectable, spatially-resolved emission lines. All exhibit LINER-AGN behaviour.

		


	\subsection{WHaN2 Plots}
		\label{subsec:WHaN2}
		\begin{figure}
			\centering
			\includegraphics[width=0.8\textwidth]{chapter5/WHaN2.png}
			\caption[WHaN2 plot for IC 4296]{WHaN2 plot for IC 4296. The colour scale (blue to yellow) represents increasing distance from the galaxy centre.}
			\label{fig:WHaN2}
		\end{figure}

		\begin{figure}
			\centering
			\includegraphics[width=0.8\textwidth]{chapter5/WHbN1.png}
			\caption[VIMOS WHbN1 plot]{VIMOS WHbN1 plot. All galaxies have decreasing H$\beta$ emission with increasing distance from the galaxy centre.}
			\label{fig:WHbN1}
		\end{figure}

		The final spatially-resolved line ratio diagnostic plot that we use is that of the equivalent width of H$\alpha$ versus $log(\text{[\ion{N}{ii}]/H\alpha})$, known as the WHaN2 plot \citep{}. In Fig.\,\ref{fig:WHaN2} we show the WHaN2 plot for the MUSE derived emission lines of IC 4296. We do not include IC 1459 and NGC 1316, as they have already been classified via the classic BPT plots (see Section \ref{subsec:BPT}). We interpret the weak AGN classification of IC 4296 as LINER-AGN.

		For the emission lines measured from the VIMOS datacubes, we use H$\beta$ as a proxy for H$\alpha$ and [\ion{N}{i}] as a proxy for [\ion{N}{ii}] utilising the Balmer, Nitrogen and stellar `decrements' to transform the classification boundaries (see Section \ref{subsec:Ndec}). We call the resulting diagnostic plot the WHbN1 plot, as shown in Fig.\,\ref{fig:WHbN1}.

		

% Comment on this


	\subsection{H$\beta$ Profiles}
		\label{subsec:Hb}

		\begin{figure}
			\centering
			\includegraphics[width=\textwidth]{chapter5/vimos/Hbeta_profile.png}
			\caption[VIMOS H$\beta$ radial profiles]{H$\beta$ flux radial profiles derived from the VIMOS datacubes. Solid red lines show $F(\mathrm{H\,\beta}) \propto r^{-2}$.\label{fig:Hb_profile_VIMOS}} 
			
		% \end{figure}

		% \begin{figure}
		% 	\centering
			\includegraphics[width=0.66\textwidth]{chapter5/muse/Halpha_profile.png}
			\caption[MUSE H$\alpha$ radial profiles]{H$\alpha$ flux radial profiles derived from the MUSE datacubes. Solid red lines show $F(\mathrm{H\alpha}) \propto r^{-2}$.\label{fig:Ha_profile_MUSE}} 
			
		\end{figure}
		
		As discussed in Section \ref{sec:ETG}, many galaxies classified as LINERs are in fact LIERs \citep[see e.g.][]{Sarzi2005, Sarzi2010, Singh2013, Belfiore2016}, i.e.\ the source of the ionizing radiation is not concentrated in the nuclear regions of the galaxies. As we can see from Figs.\ref{fig:BPT} and \ref{fig:SAURON}, most bins of a given galaxy are tightly grouped together on the BPT plots. To check if the ionizing radiation is entirely due to an AGN throughout the host galaxy, we plot the H$\alpha$ and H$\beta$ flux radial profiles for the MUSE and VIMOS data in Figs.\,\ref{fig:Ha_profile_MUSE} and \ref{fig:Hb_profile_VIMOS}, respectively. A point source, such as an AGN, will result in a profile having a $r^{-2}$ shape, where $r$ is the distance of the bin from the galaxy centre. A shallower profile points to a spatially-extended source, presumably non-circumnuclear stellar processes such as star formation or radiation from pAGB stars (see Section \ref{subsubsec:JetFeedback}), as the dominant cause of the ionization in the outer parts of the galaxy (i.e.\ LIER rather than LINER). All the galaxies with spatially-extended Balmer emission in our Southern Sample revel a central point source as the dominant source of the ionizing photons. The H$\beta$ profile of NGC 612 has a larger scatter suggesting that stellar processes also have some impact (likely radiation from pAGB stars, as these galaxies are ETGs and therefore likely have low star-formation rates).

		


	\subsection{MEx Diagnostic Plots}
		\label{subsec:MEx}
		The mass--excitation (MEx) plot of \citet{Juneau2011} is useful to classify galaxies with intermediate excitation levels ($0.5 < \mathrm{[\ion{O}{iii}]/H\,\beta} < 8.0$), particularly as it does not depend on faint lines such as the [\ion{N}{I}] doublet. We use here the calibrations by \citet{Nyland2016}, including an attempt to separate LINERs powered by AGN from those powered by pAGB stars (the former have [\ion{O}{iii}] equivalent widths $>0.8$\,\AA). We use an aperture of 3\arcsec to plot the nuclear MEx for our Southern Sample galaxies, and the resulting diagnostic plot is shown in Fig.\,\ref{fig:MEx}. 


		\begin{figure}
			\centering
			\includegraphics[width=0.8\textwidth]{chapter5/nuclear_MEx.png}
			\caption[Nuclear mass--excitation plot]{Nuclear MEx plot, allowing to classify the sources of the ionizing radiation within the central 3\arcsec. Data points derived from the VIMOS and MUSE datacubes are in red and blue, respectively. Crosses mark galaxies with [\ion{O}{iii}]$\lambda$5007 equivalent width $> 0.8$\,\AA, while filled circles show galaxies with [\ion{O}{iii}]$\lambda$5007 equivalent width $\leqslant 0.8$\,\AA. The VIMOS and MUSE data points of IC 1459 are linked with a black dashed line. Classification boundaries are from \citet{Nyland2016}.}
			\label{fig:MEx}
		\end{figure}


\section{Discussion}
	\label{sec:gasDiscussion}
	In the previous subsections we have shown that galaxies in our Southern Sample have ionized gas masses of $10^4$--$10^6\,\mathrm{M_\odot}$. These are at the upper limit of, and possibly exceed, the typical masses observed in ETGs, although given that our sample contains particularly massive galaxies, high ionized gas masses are not unexpected. 

	The only slow rotator with detected, spatially-extended ionized gas, IC 1459, shows evidence through gas--star kinematic misalignment that the gas has an external origin, although it is possible that the gas has been re-accreted after it was first ejected during the merger that resulted in the stellar KDC. 

	Of the other 3 galaxies with detected, spatially-extended ionized gas (NGC 612, NGC 1316 and NGC 3100, all fast rotators), we find that one, NGC 3100, again has kinematic evidence of an external gas origin; another, NGC 612, is consistent with an internal origin; while the situation for the last one (NGC 1316) is not clear, although is definitely not in a settled disc. This is in keeping with the findings of \citet{Davis2011a}, who showed that $36\pm5$\% of fast rotators have ionized gas kinematics misaligned with respect to the stellar kinematics, while slow rotators have a flat distribution of misalignments. This indicates that slow rotators are dominated by external sources of gas, while fast rotators can have either internal or external sources. 

	All galaxies with detectable emission lines in our Southern Sample show LINER or Seyfert 2 characteristics, and the two galaxies that are classified as Seyfert 2s (NGC 612 and NGC 3100) are very close to the boundary between the Seyfert 2 and LINER classes. Both are also fast rotators. This is consistent with the results of \citet{Nyland2016}, who showed that all ETGs in the Atlas$^\text{3D}$ sample classified as Seyferts are also fast rotators. 

	Overall, we thus find that the properties of our Southern Sample galaxies are consistent with those of radio-detected, jet-mode AGN. These are ordinary ETGs experiencing an active phase, presumably due to the central black holes currently accreting gas in some form. 
