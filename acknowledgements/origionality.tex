\begin{originalitylong}
I, Joshua Warren, carried out the work in this thesis at the Department of Astrophysics, University of Oxford between October 2014 and December 2017 under the supervision of Prof Martin Bureau. My research was funded by a Science and Technology Facilities Council (STFC) grant (Grant Code: 1577871). I declare that no part of this thesis has been submitted in support of another degree, diploma or other qualification at the University of Oxford or any other university. Except where otherwise stated, the work in this thesis is all mine in conjunction with the credited work by the rest of our research group, consisting of: Dr Isabella Prandoni and her PhD student Illaria Ruffa, Dr Robert Laing, Dr Paola Param, Dr Hans de Ruiter, Dr Arturo Mignano, Prof Martin Bureau and I. All are based at the Italian National Institute for Astrophysics (INAF) except Dr Robert Laing who is based at Jodrell Bank and Prof Martin Bureau and I who are both at University of Oxford.

We intend to shorten this thesis into journal paper form to be submitted, probably to the Monthly Notices for the Royal Astronomical Society (\mnras), for publication in early 2018, but currently no part of this thesis has been published. 

This thesis makes extensive use of observation from the Visible Multi-object Spectrograph \citep[VIMOS; ][]{LeFevre2003} and archival observations from the Multi-unit Spectroscopic Explorer \citep[MUSE; ][]{Bacon2010}, both of which are on the European Southern Observatory's (ESO's) Very Large Telescope (VLT). We also make use of observations from the Atacama Pathfinder Experiment \citep[APEX; ][]{Gusten2006} and the Atacama Large Millimeter/submillimeter Array (ALMA). ALMA is a partnership of ESO (representing its member states), NSF (USA) and NINS (Japan), together with NRC (Canada), MOST and ASIAA (Taiwan), and KASI (Republic of Korea), in cooperation with the Republic of Chile. The Joint ALMA Observatory is operated by ESO, AUI/NRAO and NAOJ.

Within this thesis we make use of data products from the Two Micron All Sky Survey, which is a joint project of the University of Massachusetts and the Infrared Processing and Analysis Center/California Institute of Technology, funded by the National Aeronautics and Space Administration and the National Science Foundation. We also make use of the SIMBAD database and Portal tool, operated at CDS, Strasbourg, France. 

Finally, this research made use of the following resources and programs: ArXiv.org; the Astrophysics Data System (ADS); \textsc{astropy}\footnote{\url{http://www.astropy.org}}, a community-developed core \textsc{python} package for Astronomy (Astropy collaboration, 2013) \citep{TheAstropyCollaboration2013}; \textsc{kinemitry} \citep{Krajnovi2006}; \textsc{ipython} \citep{Perez2007}; \textsc{matplotlib} \citep{Hunter2007}; \textsc{p3d} \citep{Sandin2010, Sandin2011}; \textsc{ppxf} \citep{Cappellari2004}; \textsc{py3d} \citep{Sanchez2011, Husemann2013, Husemann2014}; \textsc{scipy} \citep{Oliphant2007, Millman2011}/\textsc{numpy} \citep{VanderWalt2011}; \textsc{spectools} by Ryan Houghton and Simon Zieleniewski and \textsc{voronoi\_2d\_binning} \citep[including the SAURON colormaps; ][]{Cappellari2003}.

The copyright of this thesis rest with the author. No quotation from it or information derived from it may be published without acknowledgment and where appropriate, consent of its author.
\end{originalitylong}